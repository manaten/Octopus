\documentclass[a4j,12pt]{jreport}
\usepackage[dvips]{graphicx}
\usepackage{vmargin, fancyheadings, amsmath ,float, alltt}
%\usepackage{misc}
\usepackage{ascmac}
%\usepackage{here}
\usepackage{txfonts}

\newcommand\bibAnnoteFile[1]{}
\newcommand\bibAnnote[1]{\\ {\textbf 関連論文: }#1}

%\setlength{\textwidth}{150mm}       % テキストの幅
%\setlength{\textheight}{212mm}      % テキストの高さ
%\setlength{\oddsidemargin}{5mm}     % 偶数ページの左マージン
%\setlength{\evensidemargin}{5mm}    % 奇数ページの左マージン
%\setlength{\topmargin}{25.4mm}         % 上のマージン
%\renewcommand{\baselinestretch}{1.1}


\title {Webアプリケーション開発のための分散JavaScript言語}
\author{東京工業大学 \\情報理工学研究科 \\数理・計算科学専攻 \\学籍番号 09M37117\\加藤 真人\\\\指導教員\\脇田 建 准教授}
%\date {2012.2.xx (xxx)}

\begin {document}

\pagestyle {fancy}
\pagenumbering {roman}
\maketitle
\tableofcontents
\listoffigures
%\listoftables

\pagestyle {fancy}
\pagenumbering{arabic}

%------------------------------------------------------------------------------
\chapter{はじめに}
\section{背景}
\section{貢献}

\section{論文の概要}
 本論文では以下の内容を述べる.
 n章では・・・

%------------------------------------------------------------------------------
\chapter{Webアプリケーション開発}
この章では
\section{Webアプリケーションとは}
この節ではまず、Webアプリケーションとはどういうものか、実在する例を用いて解説する。

\subsection{最も単純なWebアプリケーション}
最も基本的なWebアプリケーションの開発形態は、サーバーサイドプログラムを用い、ユーザーの入力に応じて動的にWebページを書き換えるものである。

\subsection{動的に通信を行うWebアプリケーション}
ブラウザ上でJavaScriptがユーザー入力に応じて動的にサーバーサイドのプログラムと通信、ページ書き換えを行うことでよりインタラクティブなWebアプリケーションを作成することができる。


\section{開発支援}
\subsection{フレームワークを用いたWebアプリケーション}
前節の形態のWebアプリケーションが現在最も主流であるが、見てきたようにその開発は少々手間である。これらの手間を解決するために、大規模な開発ではフレームワークを用いるのが主である。ここではその内の幾つかを紹介する。

\subsection{サーバーサイドJavaScript node.js}
\subsection{WebSocket}
\subsection{多言語からJavaScriptへの変換}


%------------------------------------------------------------------------------
\chapter{分散JavaScript}
この章では
\section{設計思想}
\section{概観}
\subsection{プログラム例}
\section{仕様}
\subsection{言語を構成する要素}
\subsection{式}
\subsection{構文}
\section{ライブラリ関数}

%------------------------------------------------------------------------------ 
\chapter{分散JavaScriptの実装}
この章では分散JavaScriptをどのように既存のブラウザ・サーバー上に実現するかについて論じる。
\section{構成}
\section{コード変換}
\subsection{Rhino Astノード}
\subsection{標準形分散JavaScript}
\subsection{Continuation Passing Style}
\section{補助ライブラリ}
\subsection{リモートオブジェクト・関数}
\subsection{制御構文関数}
\section{最適化}
 
%------------------------------------------------------------------------------
\chapter{評価}
この章では
\section{主観評価}
\section{実行速度評価}
 
 
%------------------------------------------------------------------------------
\chapter{関連研究}


%------------------------------------------------------------------------------
\chapter{今後の課題}
この章では


\bibliographystyle {jalpha}
\bibliography {paper}

\end {document}
