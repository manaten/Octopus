\documentclass[a4j,12pt]{jreport}
\usepackage[dvips]{graphicx}
\usepackage{vmargin, amsmath ,float, alltt}
%\usepackage{misc}
\usepackage{ascmac}
%\usepackage{here}
\usepackage{txfonts}
\usepackage{verbatim}
\usepackage{listings,jlisting}
\renewcommand{\lstlistingname}{コード}
\lstset{
  basicstyle=\ttfamily\small,
  commentstyle=\textit,
  classoffset=1,
  keywordstyle=\bfseries,
  frame=tRBl,
  framesep=5pt,
  showstringspaces=false,
  numbers=left,
  stepnumber=1,
  numberstyle=\small,
  tabsize=2
}

\newcommand\bibAnnoteFile[1]{}
\newcommand\bibAnnote[1]{\\ {\textbf 関連論文: }#1}

%\setlength{\textwidth}{150mm}       % テキストの幅
%\setlength{\textheight}{212mm}      % テキストの高さ
%\setlength{\oddsidemargin}{5mm}     % 偶数ページの左マージン
%\setlength{\evensidemargin}{5mm}    % 奇数ページの左マージン
%\setlength{\topmargin}{25.4mm}         % 上のマージン
%\renewcommand{\baselinestretch}{1.1}


\title {Webアプリケーション開発のための分散JavaScript言語}
\author{東京工業大学 情報理工学研究科 \\数理・計算科学専攻 \\学籍番号 09M37117 加藤 真人\\\\指導教員 脇田 建 准教授}
%\date {2012.2.xx (xxx)}

\begin {document}

\pagenumbering {roman}
\maketitle
\tableofcontents
\listoffigures
%\listoftables

\pagenumbering{arabic}

% ------------------------------------------------------------------------------
\chapter{はじめに}
\section{背景}
Webアプリケーションとは、Webブラウザ上で動作するアプリケーションのことである。古くはHTMLのフォーム機能とサーバーサイドでの
HTML生成を用い、ユーザーの入力によってページを動的に切り替えるものであり、単純なインターネット掲示板などがそれに該当する。
これらのWebアプリケーションでは、ユーザーのフォームへの文字入力、及びボタン入力によって表示されるWebページの内容を変えることが
できる。
そして最近ではブラウザ上で動作す言語であるJavaScriptの使用の活発化、
及びHTML5を始めとするWebブラウザをとりまく環境の高度化により、実用的で視覚的にもリッチでかつインタラクティブな
Webアプリケーションが多数登場している。その中にはWebブラウザ上でメールの送受信を行える高機能メーラであるGmailや、
様々なデバイスで記録した情報を一挙に管理できるEvernoteなど、
Webの特性を活かすことで従来のデスクトップアプリケーションを置き換えうるようなものもある。
これらの新しいWebアプリケーションの特徴として、クライアントであるブラウザ上でJavaScriptのプログラムが動作し、
それがユーザーのボタン入力などをトリガとしてサーバーのプログラムと動的に通信を行い、得られた結果を用いて動的にビューであるHTMLを
書き換えるということがあげられる。これによって、より従来のデスクトップアプリケーションに近い、ユーザーの操作にすぐに反応し、
必要な情報を瞬時に表示できるアプリケーションとなっている。
また、FacebookやTwitterなどの、複数のユーザーが関与するアプリケーションもある。これらのアプリケーションでは、あるユーザーの
入力が他のユーザーの視覚に影響すし、ユーザーのコミュニケーションを助ける。こういったリアルタイム性のあるアプリケーションも、
昨今のWebアプリケーション事情の変化がもたらしたものであると考えられる。
このようなWebアプリケーション事情はますます複雑高度化していくものであると考えられる。

本研究では、これらWebアプリケーションの開発に注目をする。昨今のWebアプリケーションは上述のとおり、ブラウザ上で動作する
JavaScriptのプログラムと、サーバーで動作する他のプログラムがアプリケーションの実行に伴い動的に通信を行いながら動作する。
そのため、その開発もクライアント用にJavaScriptでプログラムを記述し、サーバー用にまた別の言語でプログラムを記述する必要がある。
さらにそれらを協調動作させるために、通信を意識したコーディングをする必要もある。このコーディングは本来作りたいアプリケーションの
ロジックとは異なる。Webアプリケーション開発には、このように面倒である点が存在し、プログラマの効率的な開発の妨げとなる。
そこで、これからのWebアプリケーション事情が更に発展していうく上でより効率的な開発をするための開発言語が必要であると考え、
本研究では分散JavaScript言語{\em OctopuScript}を提案する。

\section{貢献}

\section{論文の構成}
 本論文では以下の内容を述べる.
 n章では・・・

%------------------------------------------------------------------------------
\chapter{Webアプリケーションとその開発}
この章では
\section{Webアプリケーションとは}
この節ではまず、Webアプリケーションとはどういうものか、実在する例を用いて解説する。

\subsection{最も単純なWebアプリケーション}
最も基本的なWebアプリケーションの開発形態は、サーバーサイドプログラムを用い、ユーザーの入力に応じて動的にWebページを
書き換えるものである。

\subsection{動的に通信を行うWebアプリケーション}
ブラウザ上でJavaScriptがユーザー入力に応じて動的にサーバーサイドのプログラムと通信、ページ書き換えを行うことでより
インタラクティブなWebアプリケーションを作成することができる。


\section{開発支援}
\subsection{フレームワークを用いたWebアプリケーション}
前節の形態のWebアプリケーションが現在最も主流であるが、見てきたようにその開発は少々手間である。これらの手間を解決するために、
大規模な開発ではフレームワークを用いるのが主である。ここではその内の幾つかを紹介する。

\subsection{サーバーサイドJavaScript node.js}
\subsection{WebSocket}
\subsection{多言語からJavaScriptへの変換}


%------------------------------------------------------------------------------
\chapter{分散JavaScript言語 OctopuScript}
この章では
\section{設計思想}
\section{概観}

OctopuScriptは、処理系Octopusの上で動作するJavaScriptのサブセットであり、Webアプリケーションを記述するために用いる。
プログラマはOctpuScriptを用いて、従来のようにサーバー用のプログラムと、ブラウザ用のコードを記述する。従来と大きく違うのは、
それぞれのコードから他方のコードを自然に参照することができることである。例えば、以下のようなコーディングが可能になる。

\begin{lstlisting}[caption=サーバー用のOctopuScriptコード,label=ラベル]
var logging = function(logstr) {
	console.log(logstr);
};
\end{lstlisting}

\begin{lstlisting}[caption=ブラウザ用のOctopuScriptコード,label=ラベル]
server.logging("Hello from client!");
\end{lstlisting}

サーバー用のコードはサーバー起動時に、ブラウザ用のコードは各ブラウザがサーバーにアクセスした時に実行される。
このアプリケーションを動作させると、ブラウザがサーバーにアクセスするたびに、サーバーのログに''Hello, client!''という
文字列が表示される。ここで、ブラウザ用コード中のserver変数はサーバー側の(logging関数が定義されている)トップレベルを
指しているとする。
この例では、ブラウザからサーバーに定義された関数に、ブラウザで生成された文字列を引数として渡して実行しているが、

\subsection{プログラム例}
\section{仕様}
\subsection{言語を構成する要素}
\subsection{式}
\subsection{構文}
\section{ライブラリ関数}

%------------------------------------------------------------------------------
\chapter{分散JavaScript言語処理系 Octopus}
この章では分散JavaScriptをどのように既存のブラウザ・サーバー上に実現するかについて論じる。
\section{構成}
\section{コード変換}
\subsection{Rhino Astノード}
\subsection{標準形分散JavaScript}
\subsection{Continuation Passing Style}
\section{補助ライブラリ}
\subsection{リモートオブジェクト・関数}
\subsection{制御構文関数}
\section{最適化}

%------------------------------------------------------------------------------
\chapter{評価}
この章では
\section{主観評価}
\section{実行速度評価}


%------------------------------------------------------------------------------
\chapter{関連研究}


%------------------------------------------------------------------------------
\chapter{今後の課題}
この章では


\bibliographystyle {jalpha}
\bibliography {paper}

\end {document}
